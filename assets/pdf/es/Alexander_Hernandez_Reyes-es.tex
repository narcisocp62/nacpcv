\documentclass[a4paper,12pt]{article}
% xelatex
%A Few Useful Packages
\usepackage{marvosym}
\usepackage{fontspec} 					%for loading fonts
\usepackage{xunicode,xltxtra,url,parskip} 	%other packages for formatting
\RequirePackage{color,graphicx}
\usepackage[usenames,dvipsnames]{xcolor}
\usepackage[big]{layaureo} 				%better formatting of the A4 page
% an alternative to Layaureo can be ** \usepackage{fullpage} **
\usepackage{supertabular} 				%for Grades
\usepackage{titlesec}					%custom \section
\usepackage{flafter}
\usepackage{multirow}
\usepackage{float}
\usepackage{ragged2e}
\usepackage[inline]{enumitem}
\usepackage{geometry}
 \geometry{
 a4paper,
 total={170mm,257mm},
 left=5mm,
 right=5mm,
 top=15mm,
 }

%Setup hyperref package, and colours for links
\usepackage{hyperref}
\definecolor{linkcolour}{rgb}{0,0.2,0.6}
\hypersetup{colorlinks,breaklinks,urlcolor=linkcolour, linkcolor=linkcolour}

%FONTS
\defaultfontfeatures{Mapping=tex-text}
%\setmainfont[SmallCapsFont = Fontin SmallCaps]{Fontin}
%%% modified for Karol Kozioł for ShareLaTeX use
\setmainfont[
Path = ../fonts/,
BoldFont = EBGaramond-Bold.ttf,
ItalicFont = EBGaramond-Italic.ttf
]
{EBGaramond-Medium.ttf}
%%%

%CV Sections inspired by:
%http://stefano.italians.nl/archives/26
\titleformat{\section}{\Large\scshape\raggedright}{}{0em}{}[\titlerule]
\titlespacing{\section}{0pt}{3pt}{3pt}
%Tweak a bit the top margin
%\addtolength{\voffset}{-1.3cm}

%Italian hyphenation for the word: ''corporations''
\hyphenation{im-pre-se}

%-------------WATERMARK TEST [**not part of a CV**]---------------
\usepackage[absolute]{textpos}

\setlength{\TPHorizModule}{30mm}
\setlength{\TPVertModule}{\TPHorizModule}
\textblockorigin{2mm}{0.65\paperheight}
\setlength{\parindent}{0pt}

\usepackage{graphicx}

%--------------------BEGIN DOCUMENT----------------------
\begin{document}
\vspace*{-0.5cm}

\pagestyle{empty} % non-numbered pages

{\huge Alexander Hern\'andez Reyes}
\hfill
\smash{\includegraphics[width=2cm]{../photos/alex.jpg}}\\
%--------------------SECTIONS-----------------------------------
%Section: Personal Data
\section{Informaci\'on personal}
\begin{tabular}{rl}
    \textsc{Tel\'efono:}     & +598 99 270 789 \\
    %\textsc{Tel\'efono:}     & +53 55 598 315 \\
    \textsc{email:}     & \href{mailto:alex@devsysop.com}{alex@devsysop.com}, \href{mailto:alxsenen@gmail.com}{alxsenen@gmail.com} \\
    \textsc{Linkedin:}     & \href{https://www.linkedin.com/in/alxsenen/}{https://www.linkedin.com/in/alxsenen/} \\
	
\end{tabular}
\section{Resumen}
\justify
Soy una persona apasionada por la tecnología con 10 años de experiencia, administrando sistemas y servicios en Linux y Windows, también solucionando problemas rápidamente y escribiendo un plan de pasos. Soy empático, proactivo, de rápido aprendizaje, serio y responsable.
He estado trabajando con muchos equipos, mejorando mis habilidades, implementando mejores prácticas y soluciones. Siempre daré lo mejor de mí para los equipos y los clientes.

%Section: Work Experience at the top
\section{Work experience}
\justify
\begin{tabular}{p{2.9cm}| p{16cm}}
\emph{02/2022–Present} & Ingeniero Devops, Soho, Uruguay \\
\textsc{}&\footnotesize{Configuración y manejo de clusters \textbf{EKS}.}\\
\textsc{}&\footnotesize{CI/CD mediante pipelines usando \textbf{Azure Devops}.}\\
\textsc{}&\footnotesize{Despliegue y configuración de infraestructura sobre \textbf{AWS} mediante \textbf{Terraform}.}\\

\multicolumn{2}{c}{} \\  
\emph{12/2021–02/2022} & Ingeniero Devops, Inswitch, Uruguay \\
\textsc{}&\footnotesize{Trabajar con clústeres en \textbf{K8s}, \textbf{OKD}, \textbf{EKS} and \textbf{VMware}.}\\
\textsc{}&\footnotesize{Integraciones de CI/CD con canalizaciones utilizando herramientas como \textbf{Jenkins} y \textbf{Gitlab}.}\\
\textsc{}&\footnotesize{Implementaciones de automatización para minimizar los retrasos en las implementaciones.}\\

\multicolumn{2}{c}{} \\  
\emph{11/2019–02/2021} & Administrador de Sistemas y Servicios Cloud, Sonda, Uruguay \\
\textsc{}&\footnotesize{Instalaciones sobre \textbf{AWS}, \textbf{GCP}, \textbf{Azure}, \textbf{VMware}, usando sistemas como \textbf{CentOS}, \textbf{RedHat}, \textbf{openSUSE}, \textbf{Debian}, \textbf{Ubuntu}.}\\
\textsc{}&\footnotesize{Trabajo con \textbf{Docker} y \textbf{Kubernetes}, desplegando servicios para infraestructura interna y clientes externos.}\\
\textsc{}&\footnotesize{Implementación y control de la integración de sistemas con \textbf{Jenkins} y \textbf{Ansible} utilizando nuestro propio repositorio interno de \textbf{GitLab}.}\\

\multicolumn{2}{c}{} \\  
\emph{09/2018–09/2019} & Administrador de Sistemas sobre Linux, Etecsa, Cuba\\
\textsc{}&\footnotesize{Trabajo con ambientes sobre Linux Windows, en servidores o clientes.s.}\\
\textsc{}&\footnotesize{Soporte a diversas plataformas desplegadas usadas diariamente.}\\
\textsc{}&\footnotesize{Instalación de servidores y servicios mediante playbooks de Ansible.}\\
\textsc{}&\footnotesize{Manejo de controladores de dominio y aplicación de diversas politicas en la red.}\\
\textsc{}&\footnotesize{Implementaci\'on de firewalls, proxys, dns, correo y otros (Nextcloud, Owncloud).}\\

\multicolumn{2}{c}{} \\
\emph{06/2014–09/2018} & Administrador de Sistemas sobre Linux y Windows, DATYS, Cuba\\
\textsc{}&\footnotesize{Mantenimiento e implementación en diferentes ambientes como: (Windows Server 2003-2011), (Debian, Ubuntu).}\\
\textsc{}&\footnotesize{Diseño, desarrollo y despliegue de sitios webs internos y externos mediante el uso de Drupal.}\\
\textsc{}&\footnotesize{Trabajo en conjunto con el equipo de investigadores para lograr desplegar las herramientas y mejores soluciones para los diferentes proyectos.}\\

\multicolumn{2}{c}{} \\
\emph{09/2010–06/2014} & Administrador de Sistemas, Técnico de Seguridad Informática, Webmaster, UCI, Cuba\\
\textsc{}&\footnotesize{Diseño e implementación del sitio web de Seguridad Informática haciendo conciencoia en los usuarios, etc.}\\
\textsc{}&\footnotesize{Instalación de servidores para poder realizar escaneos de puertos, ficheros maliciosos, capturadores de contraseñas y virus mediante laboratorios habilitados con herramientas como Kaspersky.}\\
\textsc{}&\footnotesize{Mantenimiento, configuración e instalación de servidores sobre Windows o Linux, ej: (Windows Server 2003), (CentOS, Redhat).}\\
\end{tabular}

\section{Skills}
\justify
\begin{enumerate*}
	\item VMware, AWS, GCP, Azure \hspace{0.15cm}
	\item CentOS, RedHat, openSUSE, Debian, Ubuntu, Windows Servers \hspace{0.15cm}
	\item Jenkins, Gitlab, Ansible, Bash \hspace{0.15cm}
	\item Apache, Nginx, JBoss,  Wildfly \hspace{0.15cm}
	\item Jira, Confluence, Bitbucket \hspace{0.15cm}
	\item MySQL, PostgreSQL \hspace{0.15cm}
	\item DevOps Jr. Docker, Kubernetes \hspace{0.15cm}
	\item Troubleshooting
\end{enumerate*}

\section{Education}
\begin{tabular}{p{3cm}| p{15cm}}
	\emph{2017 - 2019}  &Ingenier\'ia en Telecomunicaciones y Electrónica, ISPJAE - Incompleto \\
	\emph{2017 - 2018}  & Administraci\'on avanzada de servicios sobre Linux, DATYS - Terminado \\
	\emph{08/2011}  & Administraci\'on de servidores Linux II, UCI - Terminado \\
	\emph{07/2010}  & Administraci\'on de servidores Linux I, UCI - Terminado \\
	\emph{2004 - 2008}  & Bachiller T\'ecnico en Electr\'onica (B.Elect.), IPTOH - Terminado \\
\end{tabular}

\section{Idiomas}
\justify
- English - Intermedio\\
- Spanish - Nativo

% \section{Proyectos}
% \justify

% 1. \href{http://ecolehavane.org}{Sitio oficial de la Alianza Francesa en la Habana} \\
% 2. \href{http://www.ciarp.org/xx}{Sitio oficial del 20 Congreso de Reconocimiento de Patrones (CIARP 2015))}\\
% 3. \href{http://www.ciarp.org/xxi}{Sitio oficial del 21 Congreso de Reconocimiento de Patrones (CIARP 2016)} \\
% 4. \href{http://acrp.cenatav.co.cu}{Asociaci\'on Cubana de Reconocimiento de Patrones (ACRP)} \\

\end{document}
